\newgeometry{
    top=15mm,
    inner=10mm,
    outer=10mm,
    bottom=15mm,
}
\selectlanguage{polish}
\topskip0pt
\vspace*{\fill}
\begin{abstract}
Niniejsza praca magisterska opisuje zagadnienie i różne metody automatycznego generowania obrazów typu Thread Art oraz opisuje autorską metodę wykorzystującą sieci neuronowe do stworzenia maski z wagami, która później zostaje użyta do wygenerowania lepszych jakościowo obrazów. Implementacja metody uwzględnia szereg parametrów, takich jak liczba wykorzystanych gwoździ oraz sposób ich rozmieszczenia, a ponadto parametryzuje właściwości tworzenia maski dając możliwość wyczulenia algorytmu na częstsze wybieranie nitek biegnących przez tło obrazu, jego obiekt lub krawędzie obiektu. W niniejszej pracy przedstawione zostanie również zagadnienie jakim jest sztuka komputerowa, sztuka algorytmiczna oraz wprowadzenie do tematyki automatycznego generowania obrazów Thread Art i przedstawienie szeregu metod wykorzystanych w innych rozwiązaniach tego problemu. Następnie przedstawiono także autorską metodę tworzenia takich obrazów, sposób implementacji tej metody, a także przedstawiono i omówiono wyniki. W podsumowaniu, oprócz oceny otrzymanych wyników, zaproponowano również dwa sposoby na kontynuowanie projektu w przyszłości. 

\end{abstract}
\keywordspl{Thread Art, grafika komputerowa, sztuczna inteligencja, algorytmy,\linebreak inżynieria oprogramowania} 

\vspace*{\fill}

\selectlanguage{english} 
\begin{abstract}
This master thesis describes concept and various methods of automatic generation of Thread Art images and also it describes the proprietary method that uses neural networks to create a mask with weights, which is then used to generate better-quality images. The implementation of this method takes into account a number of parameters, such as the number of nails used and the shape of their arrangement, and also parameterizes the properties of creating a mask, making it possible to make the algorithm sensitive to more frequent selection of threads running through the background of the image, its object or the edges of the object. This paper presents the issue of computer art, algorithmic art and an introduction to the topic of automatic generation of Thread Art images and a presentation of a number of methods used in other solutions to this problem. Then, the proprietary method of creating such images is presented along with the implementation of this method, and the results are presented and discussed. In conclusion, in addition to the evaluation of the results obtained, two ways to continue the project in the future are also proposed.
\end{abstract}
\keywords{Thread Art, computer graphics, artificial intelligence, algorithms,\linebreak software engineering}
\selectlanguage{polish}
\vspace*{\fill}
\restoregeometry